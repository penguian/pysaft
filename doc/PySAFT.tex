\documentclass[11pt]{report}
\usepackage[utf8]{inputenc}
\usepackage{amssymb,amsmath,amsthm}
\usepackage{graphicx}
\usepackage[boxed]{algorithm2e}

% Title Page
\title{PySAFT and MPIPySAFT: fast approaches to scaling up SAFT}
\author{Paul Leopardi}

\begin{document}
\maketitle

\begin{abstract}
This document describes the design of the PySAFT and MPIPySAFT prototypes.
\end{abstract}
\chapter{Introduction -- SAFT}
The SAFT program, designed and implemented by Sylvain For\^et,
produces alignment-free statistics for pairs of biological sequences,
in particular, the D2 statistic and its corresponding p-value, given certain assumptions.
Essentially, SAFT operates by taking two files, each containing a list of sequences: 
a \emph{query file} and a \emph{database file}; 
and for each query sequence, computes the D2 statistic and p-value corresponding to a pairing
with each database sequence. 
It does this by coding each query sequence and each database sequence one at a time in a hash table. 
In turn, this means that the entire list of database sequences is kept in memory,
to be used for each query sequence.

The design of PySAFT came about after I had a discussion with Sylvain For\^et on ways of speeding up SAFT.
I realised that since the D2 statistic is an inner product between to vectors of k-mer frequencies,
the whole collection of D2 statistics for each (query, database) pair can be computed as a matrix multiplication.
The following chapters describe the design of PySAFT and its parallelized version, MPIPySAFT.
\chapter{PySAFT}
\section{Description of the Code}

PySAFT uses two SciPy sparse (\verb!csr!) matrices to store the word counts for the 
query sequences and the database sequences, and then uses SciPy sparse 
matrix multiplication to calculate D2. 
So far only SAFTN (DNA alphabet) with the D2 statistic has been implemented. 

The pseudocode for PySAFT is listed as Algorithm \ref{pysaft-algorithm}.

\begin{algorithm}
\KwData{Arguments, FASTA file of query sequences, FASTA file of database sequences
}
\KwResult{A report containing, for each query sequence: query sequence name, 
and a list of database sequences, each with corresponding D2 value, p value, and 
Benjamini-Hochberg adjusted p value,
in order of increasing adjusted p value, and cut off according to arguments
} 
Parse arguments\;
Parse input and database sequences and build frequency matrices\;
Calculate d2 by matrix multiplication\;
Calculate theoretical means and variances\;
Calculate p values\;
\For{each query}{
    Sort database sequences by p value\;
    Adjust p values using Benjamini-Hochberg \;
    Cut off database sequences by number of p values and by maximum adjusted p value\;
    Print remaining database sequence names  and corresponding D2 values, p values and adjusted p values\;
}
\caption{PySAFT pseudocode.}
\label{pysaft-algorithm}
\end{algorithm}
\subsection{Key data structures}
\paragraph{Sparse frequency matrix.}
The main data structure used by PySAFT is the \emph{sparse frequency matrix}.
This is a matrix each of whose columns is a \emph{frequency vector} for a sequence.
For a given word length $k,$ a given ordered alphabet of size $d,$ 
and a sequence $s$ of letters in that alphabet, with sequence length $n,$
a frequency vector is a vector indexed by the lexicographical ordering of all of the $d^k$ possible
words in the alphabet, each entry of which is the number of occurrences of the corresponding word
in the string.

Since there are $n-k+1$ overlapping words in a string of length $n,$
for a fixed $n,$ as $k$ increases, the number of non-zero entries in the frequency vector for a word
of length $n$ stays bounded.
Also, in applications, a list of sequences taken from the same underlying text (e.g. a genome)
would be expected to yield a list of frequency vectors with similar sets of non-zero indices.
This is the heuristic justification for using a sparse matrix for the frequency matrix.

The sparse frequency matrix in PySAFT is implemented as a SciPy sparse csr (Compressed Sparse Row) matrix.
This was chosen because the csr format fits the anticipated sparsity pattern, and because the 
\newline \verb!scipy.sparse! module includes all of the functionality needed to construct these
matrices and perform arithmetic on them.
The module \verb!saftsparse!, coded in Cython, contains the functions used to create each of the query
and database sparse frequency matrices.
The main function of this module is \verb!build_dna_sparse_frequency_matrix!, whose entire code is:
\begin{verbatim}
def build_dna_sparse_frequency_matrix(
    file_name, word_len, masked=True):
    freq_list, size_list, desc_list = build_dna_frequency_lists(
                                      file_name, word_len, masked)
    return ss.hstack(freq_list ,format="csr"), size_list, desc_list
\end{verbatim}
That is, given a file name, a word length, and the Boolean parameter \verb!masked!
to indicate whether to use masking or not,
the function calls \newline 
\verb!build_dna_frequency_lists! to create a list of frequency vectors 
\newline (\verb!freq_list!)
and the corresponding lists of sequence lengths (\verb!size_list!) and sequence descriptions
(\verb!desc_list!).
The function then calls the 
\newline \verb!scipy.sparse! function \verb!hstack! to create
a sparse frequency matrix from \verb!freq_list!, and returns it along with 
\verb!size_list! and \verb!desc_list!.
\paragraph{FASTA files and sequences.}
PySAFT uses the module \verb!SeqIO! from BioPython to parse the query and database files in
FASTA format.
The \verb!SeqIO! function \verb!parse! returns an iterator 
that produces records that are BioPython \verb!SeqRecord! objects.
In PySAFT, this function is used by the 
\newline \verb!saftsparse! function \verb!build_dna_frequency_lists! 
to obtain the sequence object, length and description of each sequence in a FASTA file.
A sequence object is BioPython \verb!Seq! object.
Sequence objects are only used within 
\newline \verb!build_dna_frequency_lists! to perform masking,
if necessary.
Otherwise, the functions in PySAFT represent sequences as character strings (\verb!char *! in Cython).

\chapter{MPIPySAFT}
\section{Description of the Code}

MPIPySAFT is an MPI implementation of PySAFT, using MPI4Py. It uses two types 
of process, one scribe process, and a rectangular array of worker processes.
The worker processes are arranged by process row and process column. 
The total number of processes is controlled by the -n parameter of mpiexec. 
The number of process rows is controlled by the \verb!--mpi_rows! parameter of MPIpysaftn.py.

Invocation looks like:
\begin{verbatim}
mpiexec -n $n ./MPIpysaftn.py --input $i --database $d \
 --mpi_rows $r --wordsize $w --pmax $p --timing
\end{verbatim}
The \verb!--timing! parameter switches on debug timing.

The process rows are parallel with respect to the query file and the process 
columns are parallel with respect to the database file. 
Each worker process addresses a unique subset of pairs of sequences, $(q,d)$, where $q$ is a sequence 
from the query file and $d$ is a sequence from the database file.

The pseudocode for MPIPySAFT is listed as Algorithm \ref{mpipysaft-algorithm}.

\begin{algorithm}
\KwData{Arguments, FASTA file of query sequences, FASTA file of database sequences
}
\KwResult{A report containing, for each query sequence: query sequence name, 
and a list of database sequences, each with corresponding D2 value, p value, and 
Benjamini-Hochberg adjusted p value,
in order of increasing adjusted p value, and cut off according to arguments
} 
Parse arguments\;
Establish which process MPI thinks this is\;
Parse query and database sequences and build frequency matrices\;
Calculate d2 by matrix multiplication\;
Calculate theoretical means and variances\;
Calculate p values\;
\If{this is a worker process}{
    Determine the number of p values within this process to
      send back to the scribe process\;
}
Create a communicator for each process row\;
Get the number of p values from each worker process in 
  process row 0\;
\For{each query}{
\If{this is a worker process}{
    Sort database sequences by p value within this process\;
    Cut off database sequences by maximum number within this process\;
    Store the global database sequence indices resulting from the sorting\;
}
    Gather the global database sequence indices, and corresponding D2 values and p values 
      into the scribe process\;

\If{this is the scribe process}{
    Sort database sequences by p value\;
    Adjust p values using approximate Benjamini-Hochberg \;
    Cut off database sequences by number of p values and by maximum adjusted p value\;
    Print remaining database sequence names  and corresponding D2 values, p values and adjusted p values\;
}
}
\caption{MPIPySAFT pseudocode.}
\label{mpipysaft-algorithm}
\end{algorithm}

The scribe process is responsible for printing the result. In particular, for 
each query, the scribe process sorts the subset of p values returned from the 
relevant process row, and performs an approximate Benjamini-Hochberg 
adjustment on the p-values. The adjustment is approximate in the sense that, 
because it does not use \emph{all} of the p-values, it is impossible to ensure that 
the largest adjusted p-value contained in the subset is not larger than the 
smallest adjusted p-value not contained in the subset.

\section{Tests and Results}

Tests were conducted on the NCI Raijin cluster with:
query file: 
\newline
\verb!amel_OGSv1.0_transcript.fa! (10157 sequences)
database file: 
\newline
\verb!Homo_sapiens.GRCh37.72.cdna.all.fa! (192628 sequences).

Walltimes (P = total processes, including scribe, R = number of process rows):

\begin{center}
\begin{tabular}{|rrc|}
\hline
P  &R &Walltime (min:sec) \\
\hline
31 &1 &9:01 \\
31 &2 &7:41 \\
31 &3 &7:09 \\
63 &1 &8:14 \\
63 &2 &5:13 \\
64 &3 &4:56 \\
\hline
\end{tabular} 
\end{center}

Breakdown for $P=64, R=2,$ (seconds):

\begin{center}
\begin{tabular}{|l|rr|}
\hline
               &scribe  &worker 1 \\
\hline
Argument parse &  0.001 &  0.002 \\
Sequence parse &  7.063 & 17.412 \\
Calculate d2   &        & 20.942 \\
Means and vars &        & 53.074 \\
Calc p-values  &        & 86.797 \\
Communicator   &178.731 &  7.570 \\
Get nbr p-vals &  0.205 &  0.003 \\
Print p-values &117.090 &115.394 \\
\hline
\end{tabular} 
\end{center}

The 178+ seconds for the communicator time for the scribe process likely 
consists mostly of time waiting for the worker processes to complete the 
previous steps and synchronize.

There is a tradeoff between time spent up to the point of calculating p-values 
vs time spent printing p-values. Having more process columns speeds up the 
first steps at the cost of slowing down the printing step. This is because, 
when there are more process columns, each worker process has less database 
entries to deal with, but the scribe process has to deal with a larger subset 
of p-values, coming from the extra process columns in a row.
\newpage
Job stats for $P=64, R=2$:

\begin{verbatim}
Service Units: 5.56
NCPUs Requested: 64             NCPUs Used: 64
                                CPU Time Used: 01:14:45
Memory Requested: 256000mb      Memory Used: 41178mb
                                Vmem Used: 52868mb
Walltime requested: 00:30:00    Walltime Used: 00:05:13
jobfs request: 400mb            jobfs used: 4mb
\end{verbatim}

Job stats for $P=31, R=2$:

\begin{verbatim}
Service Units: 4.10
NCPUs Requested: 32             NCPUs Used: 32
                                CPU Time Used: 01:50:42
Memory Requested: 256000mb      Memory Used: 82869mb
                                Vmem Used: 94649mb
Walltime requested: 00:30:00    Walltime Used: 00:07:41
jobfs request: 200mb            jobfs used: 2mb
\end{verbatim}


\bibliographystyle{plain}
%\bibliography{bib}

\end{document}          
